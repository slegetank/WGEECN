\chapter{获取以及编译Emacs}
\label{chapter:E-Obtaining-and-Building-Emacs}

\guide{
\begin{itemize}
  \item 获取包
  \item 解包,编译,以及安装Emacs
\end{itemize}
}

\section{获取包}
\label{section:E-Availability-of-Packages}

本书中描述的所有软件包,除了TEX,都是自由软件基金会的GNU软件。它们的软件和其他包都可以通过匿名FTP从ftp.gnu.ai.mit.edu下的/pub/gnu路径获取。有很多镜像网站,相关信息在GNUinfo/FTP下。

如果你不能从网上下载到,或者你希望有更简单的方式,你可以从自由软件基金会订购软件的发布版本。它们有磁盘,磁带,以及光盘版本。你还可以订购许多纸质版本的GNU手册,包括一些关于Emacs的,还有附录\ref{chapter:D-Sharing-Your-Code}中提到的Textinfo手册。要知道更多信息,包括价格,可以联系FSF:

\begin{verbatim}
Free Software Foundation, Inc.
59 Temple Place - Suite 330
Boston, MA 02111-1307 USA
Telephone: +1-617-542-5942
Fax: +1-617-542-2652
Email: gnu@prep.ai.mit.ed
\end{verbatim}

本书中提到的FSF提供的包有:

\begin{description}
  \item[Emacs] 编辑器本身,包括很多Lisp扩展。源文件以emacs-x.y.tar.gz的形式存在,其中x和y分别是最新版本的主版本和小版本号(当前是19.34)。
  \item[Textinfo] GNU文档系统,包括makeinfo和一个关于如何编写Textinfo文档的手册。需要TEX来生成可打印的手册。形式为textinfo-x.y.gz。当前版本为3.7。
  \item[Emacs Lisp Reference Manual] The Emacs Lisp Reference Manual的Textinfo文档的源文件以elisp-manual-19-x.y.tar.gz的形式存在。(19是Emacs的主版本。)当前手册的版本是2.4。对于Emacs Lisp程序员来说,一个在线版本的手册是不可或缺的。
  \item[Shar untilities] 包括shar和unshar,用来创建和解包软件发布版本。形式为sharutils-x.y.gz。当前版本是4.2。shar文件不需要使用unshar解包;只需要使用标准的UNIX sh命令就可以了。
  \item[Gzip] Compression and decompression package,形式为gzip-x.y.shar。当前版本为1.2.4。
  \item[Tar] 另一个用来创建和解包软件发布版本的程序,形式为tar-x.y.shar.gz。当前版本为1.11.8。注意不像大多数其他的实现,GNU tar可以处理.tar.gz文件而并不需要gzip的存在。
  \item[The Jargon File] 在线Hacker Jargon File(在前言中引用过)在FSF中也有提供,文件为jargversion.txt.gz,当前版本为400。也有Info格式的版本,jargversion.info.gz。它是黑客知识的宝库,会以书籍的形式The New Hacker's Dictionary定期发布。
\end{description}

你可以通过TEX用户组获取到TEX。

\begin{verbatim}
TEX Users Group
1850 Union Street—Suite 1637
San Francisco, CA 94123 USA
Home page: http://www.tug.org/
Email: tug@tug.org
Fax: +1-415-982-8559
\end{verbatim}

\section{解包,编译,以及安装Emacs}
\label{section:E-Unpacking-Building-and-Installing-Emacs}

就像大多数GNU软件,Emacs的解包,编译和安装也很简单。实际上,下面的说明对于几乎所有GNU软件包都有效,而不仅仅是Emacs。

\subsection{解包}
\label{section:E-Unpacking}

如果你有一个压缩的tar文件(文件名以.tar.gz,.tar.Z,或者.tgz)并且你拥有GNU tar,执行:

\begin{minted}{sh}
tar zxvf file
\end{minted}

如果你没有GNU tar,使用这个:

\begin{minted}{sh}
zcat file | tar xvf -
\end{minted}

(你可以在Gzip包中找到zcat。)如果你使用的tar来自于SVR4的衍生版本,你可能需要把xvf替换为xvof。o使你成为展开的文件的拥有者。(否则,拥有者为tar文件的发布者---你的电脑可能并不认识他。)

如果你有一个shar文件(文件名以.sh或者.shar结尾),执行

\begin{minted}{sh}
unshar file
\end{minted}

或者简单的

\begin{minted}{sh}
sh file
\end{minted}

如果shar文件本身压缩了(.Z或者.gz),首先用\mintinline{sh}{gzip -d}解压它。

\subsection{编译和安装}
\label{section:E-Building-and-Installing}

首先在要编译的软件包的顶层目录,通过运行configure脚本来配置软件。

不同的软件包拥有不同的配置选项。使用\mintinline{sh}{./configure --help}来查看这个包的选项。Emacs的选项有:

\begin{description}
  \item[\mintinline{sh}{--with-gcc}] 使用GNU C编译器来编译Emacs。
  \item[\mintinline{sh}{--with-pop}] 编译支持Post Office Protocol(POP),有时用来收取email(对于用Emacs读取邮件的用户)
  \item[\mintinline{sh}{--with-kerberos}] 对POP使用Kerberos鉴权扩展。
  \item[\mintinline{sh}{--with-hesiod}] 使用Hesiod寻找POP服务器。
  \item[\mintinline{sh}{--with-x}] 支持X窗口。
  \item[\mintinline{sh}{--with-x-toolkit}] 更好的X窗口支持;使用工具箱部件。默认使用Toolkit,但是--with-x-toolkit=motif会使用Motif工具箱替代它。
\end{description}

你还可能希望查看configure脚本在执行的时候做了什么---这可能会有一会儿---所以你还可以使用\mintinline{sh}{--verbose}选项。下面就是我经常执行的configure:

\begin{minted}{sh}
./configure --verbose --with-x --with-x-toolkit
\end{minted}

在配置完包之后,执行make。这将会编译程序并且会花相当长的时间。

下一步,执行make check。这会执行包里提供的自检程序。

假设软件正确编译并且通过了测试,使用make install来安装它。
