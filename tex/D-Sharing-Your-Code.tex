\chapter{分享你的代码}
\label{chapter:D-Sharing-Your-Code}

\guide{
\begin{itemize}
  \item 准备源文件
  \item 文档
  \item 版权
  \item 发布
\end{itemize}
}

如果你写了一个buling buling的新Emacs模式,或者一个特性,或者一个游戏,或者其他什么,你应该秉承自由软件的精神与别人分享它,这可以通过向gnu.emacs.sources新闻组发布来实现。这个附录会向你描述分享Emacs Lisp代码的一些惯例。

\section{准备源文件}
\label{section:D-Preparing-Source-Files}

在你向世界分享你的代码之前,最好对代码进行完整的测试,修复所有你遇到过的bug。通过附录\ref{chapter:C-Debugging-and-Profiling}学习更多关于测试和调试的信息。

在代码按照你的预想工作之后,你应该向每个源文件的头部添加一个注释块来描述这个文件,它的版权(如下所示),它的作者,它的版本信息,以及其他注释。下面是一个典型的开始:

\begin{minted}{elisp}
;;; foretell.el -- predict what the user will do
;;; Copyright 1996 by Mortimer J. Hacker <mjh@mjh.net>
;;; Foretell is free software distributed under the terms
;;; of the GNU General Public License, version 2. For details,
;;; see the file COPYING.
;;; This is version 1.7 of 5 August 1996.
;;; For more information about Foretell, subscribe to the
;;; Foretell mailing list by sending a message to
;;; <foretell-request@mjh.net>.
\end{minted}

文件的最后应该有这么一行:

\begin{minted}{elisp}
;;; foretell.el ends here
\end{minted}

这会在文件用email发出的时候帮助找到文件的边界(后面可能跟着签名和其他内容)。

如果你的包含有多个文件,通常要创建一个README文件来描述这个包,里面的文件,以及如何进行安装;然后使用shar程序把所有的文件整合进一个单独的发布文件里。如果你没有shar,你可以获取到一个GNU版本的,具体参照附录\ref{chapter:E-Obtaining-and-Building-Emacs}。

\section{文档}
\label{section:D-Documentation}

最起码的,你的源文件应该在头部的注释块中有足够的注释,这样用户就能够明白这个文件的作用。如果你的代码是自解释的就更好了---也就是说,在你的函数和变量定义时恰当的使用了文档字符串。

如果你希望更严肃地编写文档,你可能会考虑为你的包创建一个Textinfo手册。Textinfo是GNU系统的标准文档格式。Textinfo文件可以通过makeinfo程序处理生成Info文件,这种树形的文件可以通过Emacs的Info模式来浏览。Textinfo文件还能通过TEX排版系统处理来生成良好格式的可打印手册。

一个很不错的关于如何编写Textinfo手册的Info手册包含在GNU textinfo包里,其中还包含着makeinfo。关于如何获取它以及TEX,参照附录\ref{chapter:E-Obtaining-and-Building-Emacs}。

\section{版权}
\label{section:D-Copyright}

你可以自由的对你的代码设定任何你想要的版权条款,当然要合法。大多数Emacs Lisp包的作者选择让他们的软件遵循GNU General Public License,一种由自由软件基金会(Free Software Foudation)发起的特殊种类的版权以使它们的软件“自由(free,在权限上,而不是价格上)”。遵循GPL的软件会被确保自由访问,但这对于某些情况并不合适,例如你要把你的软件发布到公共领域(这时,其他人可以合法的拷贝你的软件,修改它,声明这是它自己的,贩售它,并且拒绝继续分发代码)。

如果你希望让你的软件基于GPL(一种幽默的称之为使你的软件“copylefting”的过程),你需要把GPL的条款放入到你的源文件里,或者在一个单独的文件里(通常是COPYING)然后在你的每个源文件里引用它(参照本附录开头处的例子)。在Emacs中你可以通过按下\verb|M-x| describe-copying \verb|RET|来查看GPL。

\section{发布}
\label{section:D-Posting}

当你的软件已经添加完版权,做好了文档,测试并且调试过,你就可以使用你喜欢的新闻阅读器将你的软件发布到gnu.emacs.sources新闻组里了。确保为你的帖子在Subject:域添加一行有用的描述,并且确保新闻组的读者在有问题或者建议的时候知道如何联系你。注意,向gnu.emacs.sources发布任何除Emacs Lisp代码之外的东西都非常糟糕。对于非代码的发布,使用gnu.misc.discuss。
